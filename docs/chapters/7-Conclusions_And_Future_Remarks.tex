\documentclass[../main.tex]{subfiles}

\begin{document}

\section{Conclusions and Future Remarks}\label{sec:conclusions}
This dApp has successfully integrated essential features, such as Ethereum blockchain interactions, MetaMask wallet integration, and a token-based reward system. However, several limitations were identified during development that need to be addressed to enhance performance, security, and scalability.

The smart contract \texttt{Token.sol} provides a solid foundation for the application, but it could benefit from the addition of advanced features such as token burning and pausing to increase its flexibility. Additionally, both the smart contract and backend modules lack sufficient error handling and input validation, which could lead to vulnerabilities and system failures. Security improvements, including better management of private keys and more robust endpoint protection, are crucial for enhancing the dApp’s reliability.

Currently, the dApp relies exclusively on MetaMask as the wallet provider, which limits accessibility. Expanding wallet support to include other providers will make the application more inclusive for a broader user base. Another important aspect is gas optimization. Since the dApp operates on the Ethereum network, transaction costs fluctuate with network traffic. Future updates should focus on minimizing gas consumption to ensure cost efficiency, particularly during high-traffic periods.

While testing has primarily focused on basic functionality, it is important to extend the test suite to cover edge cases, performance under heavy loads, and security vulnerabilities. Comprehensive testing will ensure that the dApp can handle real-world conditions and potential threats.

One feature that remains incomplete is the NFT trading section. Once implemented, this functionality will allow users to buy and sell NFTs within the platform, expanding the scope of the dApp and aligning it with current trends in the decentralized space.

In conclusion, while the dApp demonstrates key functionalities, it requires further development to address its limitations in terms of security, wallet support, gas optimization, and feature completion. By tackling these issues, the dApp will evolve into a more robust and scalable solution, capable of meeting the growing demands of the blockchain ecosystem.

\end{document}