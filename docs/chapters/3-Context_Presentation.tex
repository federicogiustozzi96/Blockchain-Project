\documentclass[main.tex]{subfiles}
\usepackage[utf8]{inputenc}
\usepackage{amsmath}
\usepackage{graphicx}
\usepackage{hyperref}
\usepackage{enumitem}

\begin{document}

\section{Context Presentation}\label{sec:context_presentations}
The primary aim of our decentralized application (DApp) is to create an engaging and rewarding platform that combines gaming with blockchain technology. By integrating classic mini-games such as Tetris, Minesweeper, and Snake, the DApp not only provides entertainment but also incentivizes player participation through a unique token economy.

The DApp allows players to earn Donuts (DNT) tokens based on their performance in these games. These tokens can then be used to purchase exclusive NFTs, which represent unique digital assets within the platform. Additionally, players have the option to trade DNT tokens for Ethereum (ETH), thereby bridging the gap between virtual rewards and real-world value. Another significant feature of the DApp is the inclusion of quizzes related to blockchain technology. Players can earn DNT tokens by correctly answering questions on Blockchain, Smart Contracts, and Ethereum, further enhancing their engagement and knowledge.

By combining gaming with a blockchain-based reward system, the DApp aims to:
\begin{itemize}
    \item \textbf{Engage Users}: Provide an interactive gaming experience that keeps users entertained while incentivizing their participation with valuable rewards.
    \item \textbf{Educate Users}: Offer educational content in the form of quizzes to improve users' understanding of blockchain technology.
    \item \textbf{Create Value}: Enable users to earn and trade digital assets, adding real-world value to their achievements within the platform.
\end{itemize}
Overall, the DApp seeks to merge fun and functionality, creating a compelling platform that not only entertains but also educates and rewards its users.

\subsection{Why Using a Blockchain, and What Type Thereof to Use in Production}

Blockchain technology is central to the functionality and value proposition of our DApp. The decision to use a blockchain was driven by several key factors:

\begin{itemize}
    \item \textbf{Decentralization}: Blockchain provides a decentralized framework, ensuring that the game data and token transactions are not controlled by a single entity. This decentralization enhances transparency and reduces the risk of fraud or manipulation.
    \item \textbf{Security}: Blockchain's cryptographic nature ensures that transactions are secure and tamper-proof. Each transaction is recorded in a block and linked to previous transactions, making it nearly impossible to alter past records.
    \item \textbf{Transparency}: All transactions and token transfers are publicly recorded on the blockchain, allowing for full transparency. This transparency builds trust among users and ensures the integrity of the reward system.
    \item \textbf{Token Economy}: Blockchain enables the creation and management of digital tokens (DNT) that can be used within the platform. These tokens can be traded or redeemed for real-world value, providing users with tangible benefits.
\end{itemize}

When choosing the type of blockchain for production, several factors need to be considered:

\begin{itemize}
    \item \textbf{Public vs. Private Blockchains}: For our DApp, a public blockchain like Ethereum is preferred due to its widespread adoption, security features, and support for smart contracts. Public blockchains offer greater transparency and are more suitable for applications that require open and trustless interactions.
    \item \textbf{Scalability and Cost}: While public blockchains provide numerous benefits, they may face issues related to scalability and transaction costs. To address these challenges, Layer 2 solutions such as rollups or sidechains can be considered to enhance scalability and reduce costs without compromising security.
    \item \textbf{Smart Contract Support}: The chosen blockchain must support smart contracts, which are essential for automating token transactions and implementing game logic. Ethereum, with its robust support for smart contracts, is a suitable choice for this purpose.
\end{itemize}

In conclusion, the use of blockchain technology is pivotal for achieving the goals of our DApp, providing a secure, transparent, and decentralized platform. Ethereum, with its support for smart contracts and its established ecosystem, is the chosen blockchain for production, ensuring that the DApp meets its objectives effectively and efficiently.

\end{document}