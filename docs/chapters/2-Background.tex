\documentclass[../main.tex]{subfiles}
\usepackage[utf8]{inputenc}
\usepackage{amsmath}
\usepackage{graphicx}
\usepackage{hyperref}
\usepackage{enumitem}

\begin{document}
\section{Background}\label{sec:background}

Blockchain technology, originally introduced as the underlying system for Bitcoin, has evolved into a versatile and disruptive innovation. The concept of a blockchain dates back to the early 1990s when Stuart Haber and W. Scott Stornetta proposed a system to timestamp digital documents to prevent backdating or tampering \cite{HaberStornetta1991}. However, it was not until 2008, when an individual or group under the pseudonym Satoshi Nakamoto published the Bitcoin white paper, that the blockchain concept was formalized and implemented as the foundational technology for cryptocurrency \cite{Nakamoto2008}.

The rationale behind blockchain technology is to create a decentralized ledger that records transactions across a distributed network of computers. This approach enhances security and transparency by eliminating the need for a central authority and reducing the risk of single points of failure. Blockchain operates on the principle of consensus, where network participants agree on the validity of transactions through various mechanisms such as Proof of Work (PoW) or Proof of Stake (PoS). Each transaction is recorded in a "block," which is then cryptographically linked to the previous block, forming a continuous "chain" of blocks \cite{Bayer1993}. This structure ensures that once data is recorded, it cannot be altered without altering all subsequent blocks, thus providing a high level of security and integrity.

Key concepts in blockchain technology include:

\begin{itemize}
    \item \textbf{Decentralization}: Unlike traditional centralized databases, blockchain distributes data across a network of nodes, ensuring that no single entity has control over the entire database.
    \item \textbf{Consensus Mechanisms}: Techniques like Proof of Work (PoW) and Proof of Stake (PoS) are used to validate and agree on transactions within the network, maintaining data consistency and security.
    \item \textbf{Immutability}: Once data is recorded in a blockchain, it is extremely difficult to alter or delete, providing a permanent and tamper-evident record.
    \item \textbf{Smart Contracts}: Self-executing contracts with the terms of the agreement directly written into code, allowing for automated and trustless transactions between parties.
\end{itemize}

\subsection{Application Domain}

The application domain of blockchain technology extends far beyond its initial use case in cryptocurrency. Its potential to transform various industries is becoming increasingly evident as organizations explore its benefits for enhancing transparency, security, and efficiency.

In the financial sector, blockchain is revolutionizing traditional banking and payment systems by enabling faster, more secure transactions with reduced costs. Decentralized Finance (DeFi) platforms leverage blockchain to offer financial services such as lending, borrowing, and trading without intermediaries, thus democratizing access to financial resources.

Supply chain management is another area where blockchain technology is making significant strides. By providing an immutable record of each transaction and movement within the supply chain, blockchain enhances traceability and accountability. This capability helps in verifying the authenticity of goods, reducing fraud, and improving overall efficiency in logistics \cite{BlockchainHealthcareExamples2022}.

Healthcare is also experiencing transformative changes due to blockchain. The technology can securely manage patient records, ensuring that data is only accessible to authorized individuals while maintaining privacy and compliance with regulations such as GDPR and HIPAA \cite{BlockchainHealthcare2021}. Additionally, blockchain can facilitate secure sharing of medical research data, accelerating the development of treatments and fostering collaboration among researchers \cite{BlockchainHealthcareExamples2022}.

In the realm of digital identity, blockchain offers a solution to the problem of identity theft and data breaches. By providing a decentralized and secure way to manage digital identities, individuals can have greater control over their personal information and reduce the risk of unauthorized access \cite{BlockchainHealthcareExamples2022}.

Overall, the application domain of blockchain technology is vast and growing. Its ability to provide secure, transparent, and efficient solutions has the potential to disrupt and enhance various sectors, driving innovation and creating new opportunities for businesses and individuals alike.

\end{document}